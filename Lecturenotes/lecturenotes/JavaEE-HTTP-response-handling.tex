\chapter*{HTTP 响应处理编程}
\label{chp:JavaEE-HTTP-response-handling}

\section*{基本信息}
\sline
\begin{description}
\item[课程名称:] Java应用与开发
\item[授课教师:] 王晓东
\item[授课时间:] 第十周
\item[参考教材:] 本课程参考教材及资料如下:
  \begin{itemize}
  \item 吕海东,张坤 编著,Java EE企业级应用开发实例教程,清华大学出版社,2010年8月
  \end{itemize}
\end{description}

\section*{教学目标}

\sline

\begin{enumerate}
\item 掌握HTTP响应的内容,包括响应状态行、响应头、响应体。
\item 理解Java HTTP响应对象的类型及其生命周期,掌握响应对象的功能。
\item 学习并实践掌握部分响应对象方法的用法。
\end{enumerate}  

\section*{授课方式}

\sline
\begin{description}
\item[理论课:] 多媒体教学、程序演示
\item[实验课:] 上机编程
\end{description}

\newpage
\section*{教学内容}
\sline

%%%%%%%%%%%%%%%%%%%%%%%%%%%%%%%%%%%%%%%%%%%%%%%%%%%%%%%%%%%%%%
\section{HTTP响应的内容}

\subsection{HTTP响应的内容}

在Web服务器接收请求处理后,向客户端发送HTTP响应(Response)。响应的内容包括:

\begin{itemize}
\item 响应状态(Status Code)
\item 响应头(Response Header)
\item 响应体(Response Body)
\end{itemize}

\subsubsection{HTTP响应状态行} 

表明响应的状态信息,如成功、失败、错误。状态行的构成包括:{\kai\Red 版本 / 状
  态代码 / 状态消息}。

\tta{状态行例子}

\begin{verbatim}
  HTTP/1.1 200 ok
\end{verbatim}

\begin{enumerate}
\item 版本:使用的HTTP协议版本,如HTTP/1.1;
\item 状态代码:3位数字;
  \begin{itemize}
  \item 1xx: 收到请求,没有处理完。
  \item 2xx: 成功,响应完毕。
  \item 3xx:重定向,到另一个请求中去。
  \item 4xx:失败,没有请求的文档等。
  \item 5xx:内部错误,代码出现异常。
  \end{itemize}
\item 状态描述。
\end{enumerate}

\subsubsection{响应头} 

Web服务器在向客户端发送HTTP响应时也可以包含响应头,来指示客户端浏览器如
何处理响应体,主要包括响应的类型、字符编码和字节大小等信息。

\tta{常见响应头内容}

\begin{enumerate}
\item 指示HTTP响应可以接收到的文档类型集:Accept
\item 告知客户可以接收的字符集:Accept-Charset
\item 响应的字符编码集:Accept-Encoding
\item 响应体的MIME类型:Content-Type
\item 响应体的语言类型:Context-Language
\item 响应体的长度和字节数:Content-Length
\item 通知客户端到期时间:Expires
\item 缓存情况:Cache-Control
\item 重定向到另一个URL地址:Redirect
\end{enumerate}

\subsubsection{响应体} 

响应体类型由响应头确定,可以是任何类型。浏览器在处理响应体之前,会收到
响应头,根据响应头的信息,确定如何处理响应体。{\kai\Red 如响应头
  的Content-Type为PDF,则浏览器会启动PDF Reader来处理此响应体以显
  示PDF文档。}

\tta{常用响应类型}

\begin{enumerate}
\item 纯文本:text/plain
\item HTML:text/html
\item 图片:image/gif, image/jpeg
\item PDF:application/pdf
\end{enumerate}

\notice{注意}

\begin{itemize}
\item 文本类型响应要求响应头中包含MIME类型和字符编码集,使用{\hei\Red
    字符输出流}向客户端发送响应体数据;
\item 二进制数据类型响应需要在响应头中包含MIME类型,不要设置字符编码集,
  使用{\hei\Blue 字节输出流}向客户端发送响应体数据。
\end{itemize}

\section{HTTP响应对象}

\subsection{响应对象类型} 

HTTP响应对象类型为:javax.servlet.http.HttpServeletResponse。响应对象职责主要包括:

\begin{itemize}
\item 设置状态行;
\item 发送响应头 ;
\item 向Web浏览器发送HTTP响应体;
\item 控制页面的重定向,即将告知浏览器再发送一次请求。
\end{itemize}

\subsection{响应对象生命周期}

\begin{enumerate}
\item Web容器自动为每次Web组件的请求生成一个响应对象。
\item Web容器创建响应对象后,传入到doGet或doPost方法。
\item 通过响应对象向浏览器发响应。
\item 响应结束后,Web容器销毁响应对象,释放所占用的内存。
\end{enumerate}

\section{响应对象功能和方法}

\subsection{设置响应状态码} 

一般情况下,Web开发人员不需要通过编程来改变响应状态码,Web服务器会根据
请求处理的情况自动设置状态码,并发送到客户端浏览器。例如,当客户请求不
存在的URL地址时,Web服务器会自动设置状态码为404,状态消息为not found。

\subsubsection{public void setStatus(int code)}

直接发送指定的响应状态码,没有设置状态消息,只有默认的状态消息,如果无
对应状态消息则显示为空。

\subsubsection{public void setStatus(int code, String message)}

设置指定的状态码,同时设定自定义的状态消息,可以修改默认的状态消息。该
方法在Servlet 2.5后被舍弃,一般不要使用。

\subsubsection{public void sendError(int sc) throws IOException}

向客户端发送指定的错误信息码,可以是任意定义的整数。

\begin{javaCode}
  response.setCharacterEncoding("GBK");
  response.sendError(580); 
\end{javaCode}

\subsubsection{public void sendError(int sc, String msg) throws
  IOException}

向客户端发送指定的错误信息码和自定义状态消息。

\begin{javaCode}
  response.setCharacterEncoding("GBK");
  response.sendError(580, "自定义错误"); 
\end{javaCode}

\subsection{设置响应头} 

当客户端接收到响应状态为200时,浏览器会继续接收响应头信息,来确定响应体的类型和大小。

\subsubsection{public void setHeader(String name, String value)}

将指定名称和值的响应头发送到客户端。

\begin{javaCode}
  response.setHeader("Content-Type", "text/html");  
\end{javaCode}

\subsubsection{public void setIntHeader(String name, int value)}

设置整数类型的响应头的名和值。

\begin{javaCode}
  response.setHeader("Content-Length", 20);  
\end{javaCode}

\notice{注意} 实际项目中无需设定该响应头,Web服务器会自动计算并发送给浏览器。

\subsubsection{public void setDateHeader(String name,long date)}

设定日期类型的响应头,参数date为GMT格式的日期。

\subsection{设置响应头的便捷方法} 

\subsubsection{public void setContentType(String type)}

直接设置响应内容类型MIME响应头。

\subsubsection{public void setContentLength(int len)}

设置响应体长度,以字节为单位。

\subsubsection{void setCharacterEncoding(String charset)}

设置响应字符集,包括响应状态,响应头和响应体。

\subsubsection{public void setBufferSize(int size)}

设定响应体的缓存字节数。

如设定响应体缓存为4k:

\begin{javaCode}
  response.setBufferSize(4096);  
\end{javaCode}

\notice{注意}

{\kai Servlet在发送响应时,一般按照发送状态码、响应头和响应体的顺序进行,
  大的响应体缓存,可以允许Servlet有更多的时间发送状态码和响应头,这种情
  况发生在响应头和响应体同时写的情况。}编程的时候最好先把响应头全部设定
后,再发送响应体。

\subsection{响应对象方法——向客户端传送Cookie} 

\tta{public void addCookie(Cookie cookie)}

此方法功能将Cookie对象放置在响应头中,随响应内容到浏览器客户端,并保存
到客户端的PC的本地目录中。

\begin{javaCode}
  Cookie cookie01=new Cookie("userid", "9001");
  response.addCookie(cookie01);
\end{javaCode}

\subsection{响应对象方法——请求重定向} 

\tta{public void sendRedirect(String url)}

将对客户的响应重定向到新的URL上,让客户端浏览器对此URL进行请求。

{\kai 重定向到登录页面,相当于在浏览器地址栏上再输入一次URL地址,进行一次HTTP请求:}

\begin{javaCode}
  String url="../admin/login.jsp";
  response.sendRedirect(url);
\end{javaCode}

\subsection{设置响应体发送功能}

响应体即浏览器实际显示的具体内容,可以时HTML网页,也可以是其他文件格式,
由响应头的Content-Type决定。

响应体的类型主要分为两大类,即文本类型和二进制类型。{\Red 文本类型使用
  字符输出流PrintWriter的对象来实现;二进制类型由OutputStream的对象来实
  现。}

\tta{public PrintWriter getWriter()}

取得字符输出流。

\tta{public ServletOutputStream getOutputStream()}

取得二进制输出流。

\subsection{设置响应体——文本类型响应体发送编程}

\begin{enumerate}
\item 设置响应类型ContentType
  \begin{javaCode}
    response.setContentType("text/html"); //响应类型为 HTML 文档    
  \end{javaCode}
\item 设置响应字符编码
  \begin{javaCode}
    response.setCharacterEncoding("GBK"); //字符编码使用 GBK
  \end{javaCode}
\item 取得字符输出流对象
  \begin{javaCode}
    PrintWriter out = response.getWriter();
  \end{javaCode}
\item 向流对象中发送文本数据
  \begin{javaCode}
    out.println("<html><body></body></html>"); //输出文本字符
  \end{javaCode}
\item 清空流中缓存的字符
  \begin{javaCode}
    out.flush();
  \end{javaCode}
\item 关闭流
  \begin{javaCode}
    out.close();
  \end{javaCode}
\end{enumerate}

\subsection{设置响应体——文本类型响应体发送编程}

\samp{示例代码}

\begin{javaCode}
  response.setContentType("text/html; charset=gb2312");
  PrintWriter out = response.getWriter();

  out.println("<html><head>");
  out.println("</head><body>");
  out.println("hello! ");
  out.println("</body></html>");
  out.flush();
  out.close();
\end{javaCode} 

\subsection{设置响应体——二进制类型响应体发送编程}

\begin{enumerate}
\item 设置响应类型ContentType
  \begin{javaCode}
    response.setContentType("image/jpeg"); //响应类型为 JPEG 图片
  \end{javaCode}

\item 取得字节输出流对象
  \begin{javaCode}
    OutputStream out = response.getOutputStream(); //取得字节输出流
  \end{javaCode}

\item 向流对象中发送字节数据
  \begin{javaCode}
    out.println(200); //输出字节数据
  \end{javaCode}

\item 清空流中缓存的字节
  \begin{javaCode}
    out.flush();
  \end{javaCode}

\item 关闭流
  \begin{javaCode}
    out.close();
  \end{javaCode}

  {\Red\kai 注意:二进制响应编程不需要设置字符编码。}
\end{enumerate}