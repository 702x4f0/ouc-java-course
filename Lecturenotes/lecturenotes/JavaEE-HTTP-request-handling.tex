\chapter{HTTP 请求处理编程}
\label{chp:JavaEE-HTTP-request-handling}

\section*{基本信息}
\sline
\begin{description}
\item[课程名称:] Java应用与开发
\item[授课教师:] 王晓东
\item[授课时间:] 第十周
\item[参考教材:] 本课程参考教材及资料如下:
  \begin{itemize}
  \item 吕海东,张坤 编著,Java EE企业级应用开发实例教程,清华大学出版社,2010年8月
  \end{itemize}
\end{description}

\section*{教学目标}

\sline

\begin{enumerate}
\item 掌握HTTP协议的特点以及HTTP请求中包含哪些信息。
\item 理解Java HTTP请求对象的类型及其生命周期,掌握请求对象的功能。
\item 学习并实践掌握部分请求对象方法的用法。
\end{enumerate}  

\section*{授课方式}

\sline
\begin{description}
\item[理论课:] 多媒体教学、程序演示
\item[实验课:] 上机编程
\end{description}

\newpage
\section*{教学内容}
\sline

%%%%%%%%%%%%%%%%%%%%%%%%%%%%%%%%%%%%%%%%%%%%%%%%%%%%%%%%%%%%%%
\section{HTTP请求内容}


\subsection{Web工作模式}

Web通常使用{\bf\Red 请求—响应}模式。

\begin{itemize}\kai
\item 客户端(浏览器)向服务器发出HTTP请求,在HTTP请求中包含传递到服务器的数据;
\item Web服务器接收到请求,对请求进行处理。
\item Web服务器使用HTTP向客户端发送响应。
\item 客户端接收到响应后,进行显示或页面跳转。
\end{itemize}

\subsection{HTTP请求中包含的信息} 

HTTP请求中包含的信息包括两部分:{\kai\Red 请求头和请求体}。

\subsubsection{请求头}

\begin{xmlCode}
  GET /articles/news/today.jsp HTTP/1.1 
  Accept: */*
  Accept-Language: en-us 
  Connection: Keep-Alive 
  Host: localhost
  Referer: http://localhost/links.asp
  User-Agent: Mozilla/4.0 (compatible; MSIE 5.5; Windows NT 5.0)
  Accept-Encoding: gzip, deflate
  ...
\end{xmlCode}

\tta{HTTP请求头标记和说明}
  
\begin{description}\kai
\item[User-Agent] 浏览器的机器环境
\item[Accept] 浏览器支持哪些MIME数据类型
\item[Accept-Charset] 浏览器支持的字符编码
\item[Accept-Encoding] 浏览器支持哪种数据压缩格式
\item[Accept-Language] 浏览器指定的语言环境
\item[Host] 浏览器访问的主机名
\item[Referer] 浏览器是从哪个页面来的
\item[Cookie] 浏览器保存的cookie对象
\end{description}

{\kai\Red Java EE Web组件Servlet和JSP中可以使用请求对象的方法读取这些请
  求内容,进而进行相应的处理。}


\subsubsection{请求体} 

每次HTTP请求时,在请求头之后会有一个{\bf\Red 空行},接下来是请求中包含的提交数据,即{\bf\Blue 请求体}。

\ttc{\ding{182} GET请求}

无请求体,请求数据直接在请求的URL地址中,作为URL的一部分发送给Web服务器。
  
\begin{xmlCode}
  http://localhost:8080/webapp/login.do?id=9001&pass=9001
\end{xmlCode}


\begin{itemize}
\item 请求体为空,提交数据直接在URL上,作为请求头部分传输到Web服务器,
  通过URL的QueryString部分能得到提交的参数数据。
\item 此种方式对提交数据的大小有限制,不同浏览器会有所不同,
  如IE为2083字节。GET请求时数据会出现在URL中,保密性差,实际编程中要
  尽量避免。
\end{itemize}

\ttc{\ding{183} POST请求}

\begin{itemize}
\item 请求体数据单独打包为数据块,通过Socket直接传递到Web服务器端,数
  据不会在地址栏出现。
\item 可以提交大的数据,包括二进制文件,实现文件上传功能。
\item 原则上POST请求对提交的数据没有大小限制。
\end{itemize}

\section{Java EE请求对象}

\subsection{请求对象类型与生命周期} 

\tta{请求对象接口类型}

\begin{itemize}
\item Java EE规范中的通用请求对象要实现接口javax.servlet.ServeltRequest
\item HTTP请求对象要实现接口\\{\bf\Red javax.servlet.http.HttpServletRequest}
\end{itemize}

\tta{请求对象生命周期}

在Java Web组件开发中,不需要Servlet或JSP自己创建请求对象,它们
由{\bf\Blue Web容器自动创建},并传递给Servlet和JSP的服务方
法doGet和doPost,在服务处理方法中直接使用请求对象即可。

\subsection{请求对象类型与生命周期} 

\tta{请求对象创建}

每次Web服务器接收到HTTP请求时,会自动创建实现HttpServletRequest接口的
对象。在创建该对象之后,Web服务器将请求头和请求体信息写入请求对
象,Servlet和JSP可以通过请求对象的方法取得这些请求信息,继而可以取得
用户提交的数据。

\tta{请求对象销毁}

当Web服务器处理HTTP请求,向客户端发送HTTP响应结束后,会自动销毁请求对
象,保存在请求对象中的数据随即丢失。当下次请求时新的请求对象又会被创
建。

\subsection{请求对象功能与方法} 

\tta{请求对象方法一般分类}

\begin{itemize}
\item 取得请求头信息;
\item 取得请求体中包含的提交参数数据,包含表单元素或地址栏URL的参数;
\item 取得客户端的有关信息,如请求协议、IP地址和端口等;
\item 取得服务器端的相关信息,如服务器的IP等;
\item 取得请求对象的属性信息,用于在一个请求的转发对象之间传递数据。
\end{itemize}

\subsection{取得请求头} 

\begin{itemize}
\item String getHeader(String name)
  \begin{javaCode}
    String browser = request.getHeader("User-Agent");
  \end{javaCode}

\item int getIntHeader(String name)
  \begin{javaCode}
    int size = request.getIntHeader("Content-Length");
  \end{javaCode}

\item long getDateHeader(String name)
  \begin{javaCode}
    long datetime = request.getDateHeader("If-Modified-Since");
  \end{javaCode}

\item Enumeration getHeaderNames()
  
  \begin{javaCode}
    for (Enumeration e = request.getHeaderNames(); e.hasMoreElements(); ) {
      String headerName = (String) e.nextElement();
      System.out.println("Name = " + headerName);
    }
  \end{javaCode}
\end{itemize}

\subsection{取得请求中包含的提交参数数据} 

\begin{itemize}
\item 在Web开发中,用户通过表单或URL参数将客户端数据提交到服务器端,
  这些数据被Web服务器自动封装到请求对象中。
\item Web组件Servlet和JSP可以通过请求对象获得用户提交的数据。
\end{itemize}
  
\subsubsection{String getParameter(String name)}

取得表单或URL参数中指定名称的数据值。

\ttc{表单}

\begin{xmlCode}
  Product Name: <input type="text" name="productName" />
\end{xmlCode}

\ttc{URL参数}

\begin{xmlCode}
  productSearch.do?productName=Acer    
\end{xmlCode}

以下代码可以取得以上的参数名为productName的数据:

\begin{javaCode}
  String productName = request.getParameter("productName");
\end{javaCode}


\subsubsection{String[] getParameterValues(String name)}

取得指定名称的参数数据数组,如复选框和复选列表。

\ttc{表单复选数据}

\begin{xmlCode}
  爱好:
  <input type="checkbox" name="behave" value="旅游">旅游
  <input type="checkbox" name="behave" value="读书">读书
  <input type="checkbox" name="behave" value="体育">体育
\end{xmlCode}

如下代码取得选定的爱好:

\begin{javaCode}
  String[] behaves = request.getParameterValues("behave");
  for (int i = 0; i < behaves.length; i++) {
    out.println(behaves[i]);
  }
\end{javaCode}

\subsubsection{Enumeration getParameterNames()}

取得所有请求的参数名称。

\begin{javaCode}
  for (Enumeration enums = request.getParameterNames(); enums.hasMoreElements();) {
    String paramName = (String) enums.nextElement();
    System.out.println(paramName);
  }
\end{javaCode}

\subsubsection{Map getParameterMap()}

取得所有请求的参数名和值,包装在一个Map对象中,可以使用这个对象同时取得
所有的参数名和参数值。

\begin{javaCode}
  Map params = request.getParameterMap();
  Set names = params.keySet();
  for (Object o: names) {
    String paramName = (String) o;
    out.print(paramName + " = " + params.get(paramName) + "<br/>");
  }
\end{javaCode}

\subsubsection{ServletInputStream getInputStream() throws IOException}

取得客户端的输入流。

\begin{itemize}
\item 当用户提交的数据中包含文件上传时,提交的数据可以以二进制编码方
  式提交到服务器。
\item 当表单既有文本字段还有文件上传时,就需要对此二进制流进行解析,
  从而分离出文本和上传文件。
\item 可以使用第三方框架/Jar包实现上传文件的处理,如{\kai\Blue Apache的Common
    upload组件}。
\end{itemize}

\notice{注意}

{\kai 当使用getParameter()方法后,就无法使用getInputStream()方法,反之亦然。}

\begin{xmlCode}
  <form action="" method="" enctype="multipart/form-data">
    姓名:<input type="text" name="name"/><br/>
    照片:<input type="file" name="photo"/><br/>
    <input type="submit" value="提交"/>
  </form>
\end{xmlCode}

\subsection{取得其他客户端信息} 

\ttc{String getRemoteHost()} 取得请求客户的主机名。

\ttc{String getRemoteAddr()} 取得请求客户端的IP地址。

\ttc{int getRemotePort()} 取得请求客户的端口号。

\ttc{String getProtocol()} 取得请求的协议。

\ttc{String getContentType()} 取得请求体的MIME内容类型。

\ttc{int getContentLength()} 取得请求体为二进制流时请求体的长度。

\ttc{String getProtocol()} 取得请求的协议,一般为HTTP。

\subsection{取得服务器端信息} 

\ttc{String getServerName()} 取得服务器的HOST,一般为IP地址。

\ttc{int getServerPort()} 取得服务器接收端口。

