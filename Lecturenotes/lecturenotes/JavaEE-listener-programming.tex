\chapter{Java EE监听器编程}
\label{chp:JavaEE-listener-programming}

\section*{基本信息}
\sline
\begin{description}
\item[课程名称:] Java应用与开发
\item[授课教师:] 王晓东
\item[授课时间:] 第十二周
\item[参考教材:] 本课程参考教材及资料如下:
  \begin{itemize}
  \item 吕海东,张坤 编著,Java EE企业级应用开发实例教程,清华大学出版社,2010年8月
  \end{itemize}
\end{description}

\section*{教学目标}

\sline

\begin{enumerate}
\item 理解监听器的概念。
\item 掌握Java EE的监听器的主要功能和包含的类型。
\item 掌握监听器的编程和配置,学会编写监听器代码。
\end{enumerate}  

\section*{授课方式}

\sline
\begin{description}
\item[理论课:] 多媒体教学、程序演示
\item[实验课:] 上机编程
\end{description}

\newpage
\section*{教学内容}
\sline
%%%%%%%%%%%%%%%%%%%%%%%%%%%%%%%%%%%%%%%%%%%%%%%%%%%%%%%%%%%%%%

\section{监听器概述}

监听器,顾名思义就是能监测其他对象活动的对象,当被监测的对象发生变化时,
会自动触发运行监听器方法,完成特定的功能和任务。Java EE规范在Servlet
2.3中引入了监听器(Listener)规范。

Java EE监听器能够检测Web应用的关键对象包括:

\begin{itemize}
\item ServletContext上下文对象;
\item HttpSession会话对象;
\item ServletRequest请求对象。
\end{itemize}

监听器通常用于以下应用场景:

\begin{itemize}
\item {\hei 网站访问人数或次数计数器} 访问人数计数是所有综合门户网站的
  生命,是网站广告标价的基础。国内知名门户网站如网易、新浪等广告价格高,
  就在于其每日巨大的访问量所致。
\item {\hei 网站登录用户人数和在线用户监测} 可以使用监听器完成Web应用已
  经登录人数和在线用户列表的登记处理。这是许多网上论坛,网上购物网
  站,Mail在线系统,即时通讯系统等必须具有的功能。
\item {\hei 日志记录} 对Web应用关键的事件进行记录,如Web服务器的启动和停
  止,用户的登录和注销等事件的登记,便于日后进行系统中追踪和维护。
\item {\hei 会话超时后的清理工作}
\end{itemize}


\section{Java EE监听器类型}

Java EE提供的监听器类型包括:

\begin{enumerate}
\item ServletContext对象监听器
\item ServletContext对象属性监听器
\item HttpSession对象监听器
\item HttpSession对象属性监听器
\item HttpServletRequest对象监听器
\item HttpServletRequest属性监听器
\end{enumerate}

相关监听器接口和监听器事件如表\ref{tab:listener-and-event}所示。

\begin{table}[!htbp]
  \centering
  \caption{Java EE监听器接口和监听器事件}
  \label{tab:listener-and-event}
  \begin{tabular}{l|c|l}
    监听器接口 & 引入版本 & 监听器事件\\
    ServletContextListener & 2.3 & ServletContextEvent\\
    ServletContextAttributeListener & 2.3 & ServletContextAttributeEvent\\
    HttpSessionListener & 2.3 & HttpSessionEvent\\
    HttpSessionActivationListener & 2.3 & HttpSessionEvent\\
    HttpSessionBindingListener & 2.3 & HttpSessionBindingEvent\\
    ServletRequestListener & 2.4 & ServletRequestEvent\\
    ServletRequestAttributeListener & 2.4 & ServletRequestAttributeEvent\\
  \end{tabular}
\end{table}


\section{ServletContext对象监听器}

ServletContext对象监听器能监听Web应用上下文对象的创建和销毁这二个关键的
状态,并分别提供了不同的方法来实现对该对象创建和销毁的监听和处理。

\begin{itemize}
\item JavaEE规范定义了ServletContext监听器接
  口javax.serlvet.ServletContextListener来规范监听器的编写。
\item 提供了ServletContext监听器事件
  类javax.servlet.ServletContextEvent。
\item 定义了取得监听对象本身的方法。
\end{itemize}


ServletContext对象监听器编程方法如下:

\begin{enumerate}\kai
\item 定义监听器类并实现javax.servlet.ServletContextListener接口。
\item 实现接口中定义的如下两个方法:
  \begin{itemize}
  \item public void contextInitialized(ServletContextEvent event) \\对象创建监听方法
  \item public void contextDestroyed(ServletContextEvent event) \\对象销毁监听方法
  \end{itemize}
\item ServletContextEvent类型对象中封装了ServeltContext对象,可以用于实现对Web应用信息的访问。
\end{enumerate}

以下给出ServletContext对象监听器的一个示例。

\samplecode{监听器实现}

\begin{javaCode}
  public class ApplicationStatusListener implements ServletContextListener {
    ...
    public void contextInitialized(ServletContextEvent event) {
      ServletContext application = event.getServletContext();
      application.setAttribute("userOnlineNumber", userOnlineNumber);
      WebVisitNumber = UserOnline.getVisitNumber();
      application.setAttribute("WebVisitNumber", WebVisitNumber);
    }
    public void contextDestroyed(ServletContextEvent event) {
      ServletContext application = event.getServletContext();
      WebVisitNumber = (Integer) application.getAttribute("WebVisitNumber");
      UserOnline.saveVisitNumber(userOnlineNumber);
    }
  }  
\end{javaCode}

\begin{itemize}\kai
\item 监听器监测到ServletContext对象创建,自动执行contextInitialized方
  法,取得以前保存的网站访问次数,将历史访问次数存入ServletContext对象
  中。将来监测到用户访问网站时,自动进行计数器累加。
\item 监听器检测到ServletContext对象要销毁,Web容器自动执
  行contextDestroyed方法,完成从ServletContext对象中取得累计访问次数并
  写入数据库中。
\end{itemize}


接下来,在/WEB-INF/web.xml中配置监听器。

\samplecode{监听器配置}

\begin{xmlCode}
  <!-- ServletContext对象创建和销毁监听器 -->
  <listener>  
    <listener-class>org.javaee.ApplicationStatusListener</listener-class>
  </listener>  
\end{xmlCode}

\notice{注意}

{\bf\Red 监听器的配置不需要映射地址,也无需初始化参数,只需要监听器的包和类名即可。}

\section{ServletContext对象属性监听器}

ServletContext对象属性监听器用于监控Web应用上下文对象属性的变
化。Servlet作为Web应用级共享容器,可以使用如下操作ServletContext属性的
方法进行共享数据的保存、读取和删除:

\begin{itemize}
\item public void setAttribute(String name, Object value);
\item public Object getAttribute(String name);
\item public void removeAttribute(String name);
\end{itemize}

为监控ServletContext对象属性的变化,Java EE Servlet API提供了监听器接
口javax.servlet.ServletContextAttributeListener和事件
类javax.servlet.ServletContextAttributeEvent,接口方法包括:

\begin{itemize}\small
\item public void attributeAdded(ServletContextAttributeEvent event);
\item public void attributeRemoved(ServletContextAttributeEvent event);
\item public void attributeReplaced(ServletContextAttributeEvent event);
\end{itemize}

\tta{ServletContext对象属性监听器事件}

javax.servlet.ServletContextAttributeEvent对象是ServletContextEvent的子
类,其主要方法包括:

\begin{itemize}
\item 继承了父类方法public ServletContext getServletContext()取得
  ServletContext对象的实例;
\item public String getName() 取得发生变化的属性的名称;
\item public Object getValue() 取得发生变化的属性的值。
\end{itemize}

以下给出ServletContext对象属性监听器的一个示例。

\ttb{\ding{182} 监听器编写}

\begin{javaCode}
  public class ApplicationAttributeListener implements
  ServletContextAttributeListener {
    private String name = null;
    private String value = null;
    
    public void attributeAdded(ServletContextAttributeEvent event) {
      name = event.getName();
      value = (String) evnet.getValue();
      System.out.println("New Attribute added: " + "name" + "=" + "value");
    }

    // Other Methods.

  }
\end{javaCode}

\ttb{\ding{183} 监听器配置}

\begin{xmlCode}
  <!-- ServletContext对象创建和销毁监听器 -->
  <listener>
    <listener-class>org.javaee.ApplicationAttributeListener</listener-class>
  </listener> 
\end{xmlCode}
